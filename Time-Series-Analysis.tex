% style setup
% -----------------------------------------------------------------------
\documentclass[10pt,landscape]{article}
\usepackage{amssymb,amsmath,amsthm,amsfonts}
\usepackage{multicol,multirow}
\usepackage{calc}
\usepackage{ifthen}
\usepackage[landscape]{geometry}
\usepackage[colorlinks=true,citecolor=blue,linkcolor=blue]{hyperref}


\ifthenelse{\lengthtest { \paperwidth = 11in}}
    { \geometry{top=.5in,left=.5in,right=.5in,bottom=.5in} }
	{\ifthenelse{ \lengthtest{ \paperwidth = 297mm}}
		{\geometry{top=1cm,left=1cm,right=1cm,bottom=1cm} }
		{\geometry{top=1cm,left=1cm,right=1cm,bottom=1cm} }
	}
\pagestyle{empty}
\makeatletter
\renewcommand{\section}{\@startsection{section}{1}{0mm}%
                                {-1ex plus -.5ex minus -.2ex}%
                                {0.5ex plus .2ex}%x
                                {\normalfont\large\bfseries}}
\renewcommand{\subsection}{\@startsection{subsection}{2}{0mm}%
                                {-1explus -.5ex minus -.2ex}%
                                {0.5ex plus .2ex}%
                                {\normalfont\normalsize\bfseries}}
\renewcommand{\subsubsection}{\@startsection{subsubsection}{3}{0mm}%
                                {-1ex plus -.5ex minus -.2ex}%
                                {1ex plus .2ex}%
                                {\normalfont\small\bfseries}}
\makeatother
\setcounter{secnumdepth}{0}
\setlength{\parindent}{0pt}
\setlength{\parskip}{0pt plus 0.5ex}
% -----------------------------------------------------------------------
% -----------------------------------------------------------------------
% -----------------------------------------------------------------------

\title{Quick Guide to LaTeX}

\begin{document}

\footnotesize

% -----------------------------------------------------------------------
% -----------------------------------------------------------------------
% -----------------------------------------------------------------------



\begin{center}
     \Large{\textbf{Time Series Analysis}} \\
\end{center}



\section{What is time series}
A collection of stochastic random variables \textbf{indexed by time}, their stochastic \textbf{distributions are similar but not the same}, the data are \textbf{correlated across time}\\
In plain words, time series refers to \textbf{a series of data points}, for which they follow some kind of distribution.\\
\textbf{The goal for studying time series}, is really to investigate in those data points, and try to figure out its \textbf{underlying distribution, and internal correlation} 
\subsection{Important Characteristics}
\begin{enumerate}
	\item Trend
	\item Seasonality
	\item Periodicity
	\item Cyclical Trend
	\item Heteroskedasticity
	\item Dependence
\end{enumerate}

\subsection{Presentation \& Basic Decomposition}
$Y_t, \text{ in which t indexes time}$\\
$Y_t = m_t+s_t+x_t$$\ ,\  with : 
	\\m_t \text{ --- trend component}, 
	\\s_t \text{ --- seasonal component, seasonality means that it is a periodicity movement, it has a circle of appearance, within the circle, the overall trend should be zero }. 
	\\x_t \text{ --- stationary component}$\\
	



\begin{multicols}{3}
\setlength{\premulticols}{1pt}
\setlength{\postmulticols}{1pt}
\setlength{\multicolsep}{1pt}
\setlength{\columnsep}{2pt}

	
\section{Estimate the trend}
\begin{enumerate}
	\item Moving Average: \\estimate the trend with a moving window
	\item Parametric Regression \\ fit in a polynomial regression to estimate
	\item Non-Parametric Approach
	\begin{itemize}
	\item Kernel Regression
	\item Local Polynomial Regression
	\item Other Approaches 
\end{itemize}

\texttt{
Example: Temperature Data\\
1. Read the data file in R\\
2. Visualize the data (Avg Temp)\\
3. Estimate the trend (Moving Average)\\
4. Estimate the trend (Parametric)\\
5. Estimate the trend (Non-Parametric)\\
6. Comparison
}

\end{enumerate}


\section{Estimate the Seaonality}
General Approach: estimate and subtract $m_t$ and $s_t$\\

\begin{enumerate}
	\item Seasonal Average
			\\ $\hat s_k = w_k - \frac{1}{d}\sum_{j=1}^d w_j$\\
			\\ $k$: the seasonal group, let's say it has d groups. If the seasonality is monthly, then d = 12
			\\ $w_k$: the average of all seasonal group value
			
	\item Parametric Regression

\texttt{
Example: Temperature Data\\
1. Read the data file in R\\
2. Visualize the data (Avg Temp)\\
3. Estimate the trend (Moving Average)\\
4. Estimate the trend (Parametric)\\
5. Estimate the trend (Non-Parametric)\\
6. Comparison
}

\end{enumerate}

\section{Estimate the Stationary}


\end{multicols}



% style end
% -----------------------------------------------------------------------
\end{document}
% -----------------------------------------------------------------------
